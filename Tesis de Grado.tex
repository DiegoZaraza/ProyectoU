\documentclass [12pt,letterpaper,oneside]{book}
\usepackage [utf8]{inputenc}
\usepackage [spanish]{babel}
\usepackage {graphicx}
\usepackage [left=4.00cm, right=2.00cm, top=3.00cm, bottom=3.00cm]{geometry}
\author {Diego Andres Zaraza Toro}
\title {Diseño de arquitectura para un sistemas de evaluación y diagnóstico de niños con problemas vestibulares utilizando machine learning}
\begin{document}
    \begin{center}
    	\thispagestyle{empty}
    	\vspace*{0cm}
        \begin{figure}
            \centering
            \includegraphics[width=4cm]{Imagenes/Figuras/00}
        \end{figure}
        \huge {Diseño de arquitectura para sistema de evaluación y diagnóstico de niños con problemas vestibulares utilizando machine learning}\\[3.5cm]
        \large{Diego Andres Zaraza Toro}\\[3.5cm]
        \large Universidad Distrital Francisco José de Caldas\\
        Especialización en Ingeniería de Software\\
        Bogotá, Colombia\\
        2019\\
    \end{center}
    \newpage
    \begin{center}
        \thispagestyle{empty}
        \vspace*{0cm}
        \large{Diseño de arquitectura para sistema de evaluación y diagnóstico de niños con problemas vestibulares utilizando machine learning}\\[3.5cm]
        \large\textbf{Diego Andres Zaraza Toro}\\[3.0cm]
        \small Tesis o trabajo de grado presentada como requisito parcial para optar al título de:\\
        \textbf{Especialista en Ingeniería de Software}\\[2.5cm]
        \small Director(a):\\
        \textbf{Ph.D., Joaquin Javier Meza Alvarez}\\[2.0cm]
        \large Universidad Distrital Francisco José de Caldas\\
        Especialización en Ingeniería de Software\\
        Bogotá, Colombia\\
        2019\\
    \end{center}

    \newpage
    \section*{INTRODUCCIÓN}
    En el desarrollo de los niños se puede encontrar diversos inconvenientes que afectan su desarrollo y a su vez estos deben ser tratados por profesionales dándole el desarrollo adecuado al niño para su crecimiento y fortalecimiento de diferentes aspectos para su vida, es así como se encuentran diversos casos acerca de niños con problemas vestibulares, lo cual puede repercutir no solo en la primera infancia sino en todas las etapas de su vida, por estos que se requiere una atención de calidad donde se pueda realizar un diagnóstico acertado del problema sino también se encuentren tratamientos adecuados.
    Con el siguiente proyecto se busca dar un aspecto de calidad a las diferentes pruebas que se realizan para generar un diagnostico acertado de la problemática que pueda tener un niño en cuanto a la problemática planteada y surjan alternativas para un tratamiento adecuado que le den mejor calidad de vida.
    \newpage
    \tableofcontents
    \newpage
    \listoffigures
    \newpage
    \part{Contextualización de la Investigación}
    	\chapter{Descripción de la Investigación}
\section{Planteamiento del problema}
	La  integración  sensorial  es  un  área  fundamental  en  terapia  ocupacional  la cual trabaja  todo  lo  referente  a  la  parte  sensorio-motora  especialmente  en  niños  de diferentes edades,  esto  permite  observar  el  comportamiento de  los  niños  a  cierta edad,  por  ejemplo  según  las  etapas  de  desarrollo  normal  de  un  niño,  la  edad adecuada para aprender a anudarse los zapatos es a los 6 años pero en ocasiones algunos niños se les dificulta y pueden tardarse más tiempo en aprender, este es un síntoma para percibir que el niño tiene retrasos en su desarrollo por lo cual es necesario la asistencia de especialistas en terapia ocupacional que le den base a los niños para aprender a realizar esta labor por sus propios medios, así mismo se presentan  problemas  con  diferentes  actividades  que  en  algún  momento  deberían aprender y que son base para su desarrollo además de que les servirá para el resto de sus vidas.\\
	Cuando un terapeuta realiza la valoración de su paciente debe realizar una serie de actividades  donde  recree  diferentes  etapas  las  cuales dan  un  diagnóstico  de  que problemática o retraso puede estar presentando un niño, con esto genera una base para realizar una serie de tratamientos que ayudarían en el proceso de desarrollo de los niños, la problemática se encuentra en que estos son procesos largos para un profesional ya que existen muchas formas de tratar los retrasos a su vez se debe tener  en  cuenta  que  estos  deben  ser  dinámicos  para  que  los  niños  no  entren  en aburrimiento con los tratamientos que se aplican, a su vez los tiempos para generar un diagnostico claro pueden llevar hasta dos o tres semanas tiempo en que el niño puede aprovecharlo para avanzar en un tratamiento.\\
	Para  dar  una  solución  a  la  problemática  mencionada  se  requiere  un  prototipo  de software en el cual se pueda realizar el registro del diagnóstico y este a través de redes  neuronales pueda  generar  sugerencias  para  el  tratamiento  de  los  niños reduciendo  los  tiempos  que  este  le  tomaba  a  los  terapeutas  que  realizaban  el proceso de valoración.
\section{Objetivos de la investigación}
	\subsection{Objetivo General}
	Desarrollar prototipo de software para la evaluación y diagnosticó de niños con problemas vestibulares utilizando redes neuronales para el modelado de sistemas dinámicos.
	\subsection{Objetivos Específicos}
	\begin{itemize}
		\item Elaborar un sistema de diagnóstico para identificar los posibles problemas vestibulares que tiene un niño al ser valorado.
		\item Elaborar un sistema de tratamientos los cuales sean adecuados para tratar un problema vestibular en los niños y ver su evolución
		\item Elaborar comparativas entre los tratamientos generados y tratamientos establecidos de manera manual para ver la evolución de los niños y cual ha servido de manera más efectiva.
	\end{itemize}
\newpage
\section{Justificación de la investigación}
	\subsection{Justificación teórica}
		El vértigo y los trastornos del equilibrio son patologías que en algún momento de la etapa de crecimiento puede afectar a un niño y la cual pueden a su vez estar presentes para toda la vida, es por esto por lo que cuando un niño se marea fácilmente cuando se viaja en carro o en otro medio de transporte es necesario tratarlo para darle un mejor desarrollo a su sistema vestibular y que a futuro pueda tener mejor calidad de vida.\\
		De allí se obtiene que la presente investigación tenga como objeto de estudio el uso de redes neuronales para crear alternativas de tratamiento en los niños que pueda dar una alternativa en la búsqueda de estas ya que no convierte los tratamientos en algo limitado, sino que expande la noción y el uso que le pueda dar un terapeuta en pro del beneficio del niño en sus etapas de desarrollo.\\
\section{Hipótesis de trabajo}
	Desarrollar el prototipo permitirá optimizar y realizar una mejor labor al momento de atender a niños con problemas vestibulares dando más alternativas de tratamientos que ayudaran al terapeuta a tratar los problemas que tenga y al final se puede ver la mejora generada por los tratamientos ejecutados ya que el aplicativo tendrá la capacidad de buscar la mejor alternativa para un tratamiento efectivo.\\
\section{Marco referencial}
	\subsection{Marco Teórico}
		\subsubsection{Teoría de Integración Sensorial:}
			La Dra. Jean Ayres, (1972) fue una de las precursoras de la teoría de IS, quien logra explicar la íntima relación que existe entre el sistema nervioso y la conducta, puesto que, el proceso de IS es soberanamente adaptativo y entrelaza información proveniente de diferentes canales sensoriales para descubrir mejor, reconocer y reaccionar a los acontecimientos del ambiente, de ahí que, la IS es crítica para la percepción y la conducta. \\
			De acuerdo con lo anterior, se establecen cinco (5) posturas que aportan a la relación entre el sistema nervioso y la conducta:
			\begin{itemize}
				\item La reacción del sistema nervioso ante las situaciones de las que es participe el ser humano, se le conoce como neuroplasticidad.
				\item El cerebro actúa como un conjunto jerárquico integrado por niveles superiores que adquieren el control y son controlados por las funciónes proporcionadas a cada nivel.
				Existe una secuencia evolutiva que da como resultado la interacción entre maduración cerebral y acumulación de experiencias sensoriales.
				\item La disposición del cerebro y la conducta adaptativa son interactivas.
				\item Los individuos conservan impulso interno para cumplir retos en actividades sensoriales motoras.
			\end{itemize}
			El desarrollo de integración sensorial se considera como un aspecto de carácter automático e inconsciente que permite entender el continuo del procesamiento sensorial. El cual tiene como objetivo entender el funciónamiento del procesamiento cerebral desde el inicio: entrada-integración-salida / entorno-cerebro-conducta. \\
			Ayres, crea el siguiente esquema para mostrar que las sensaciones son la entrada primaria de información, para luego convertirse en una representación corporal que a su vez da respuesta a una actividad con un propósito especifico; lo cual favorece la constitución de funciónes integrales en el cerebro y finalmente una especialización natural de los dos lados del cuerpo y del cerebro. Todo lo anterior con el fin de precisar un aprendizaje significativo entre el sentido y la interacción con el medio.\\
			\begin{figure}
				\centering
				\includegraphics[width=10cm]{Imagenes/Figuras/01.png}
				\caption{Los sentidos, integración de sus entradas y su producto fina}
			\end{figure}
			Alteraciones o problemas del desarrollo
			Cuando se hace referencia a desarrollo psicomotor normal se habla de un proceso que permite al niño adquirir habilidades adecuadas para su edad. No obstante, como se mencionó, existe gran variabilidad en la edad en la adquisición o alcance de diferentes habilidades. Esto es relevante porque da cuenta de la dificultad de establecer claramente un límite entre lo "normal" y lo "patológico". En general, ambas esferas son diferenciadas con criterios de normalidad estadística bajo los términos desvío, significación y promedio. Así Poo Argüelles planteó “Que lo patológico es apartarse de una manera significativa de lo esperado para la edad, en un área concreta o en la globalidad e Illingworth sostuvo lo único que se puede decir es que cuanto más lejos del promedio se encuentre un niño, en cualquier aspecto, es menos probable que sea normal”. En esta perspectiva, cuando el DPM presenta características peculiares o diferentes a la "norma", se está en presencia de alteraciones o problemas del desarrollo. ¿Pero cuán apartado de la norma debe estar el DPM para ser considerado patológico? En general es sencillo estar de acuerdo en lo "muy patológico", pero no tanto cuando se intentan definir ciertas alteraciones o trastornos, que pueden discurrir entre ambos extremos.\\
			El DPM puede presentar variantes o alteraciones diversas. El retraso psicomotor, los diferentes tipos de trastornos del desarrollo y los problemas inaparentes del desarrollo son ejemplos de este tipo de alteraciones. El retraso psicomotor es uno de los cuadros más frecuentemente detectados en niños pequeños. Narbona y Schlumberger lo definieron como “un diagnóstico provisional, en donde los logros del desarrollo de un determinado niño durante sus primeros tres años de vida aparecen con una secuencia lenta para su edad y/o cualitativamente alterada”. El término retraso psicomotor, entonces, se suele mantener hasta que pueda establecerse un diagnóstico definitivo a través de pruebas formales (Accardoy otros, 2001). Alvarez Gómez et al sostienen que “debido a que es un término muy indefinido, no debería utilizarse más allá de los tres a cinco años de edad del niño, cuando ya se pueden realizar tests que miden la capacidad intelectual”.\\
			En España el término retraso psicomotor se utiliza como sinónimo de retraso del desarrollo, mientras que en América Latina es más frecuente el término retraso madurativo. Álvarez Gómez et al, por otra parte, definen al retraso del desarrollo como una demora o lentitud en la secuencia normal de adquisición de los hitos del desarrollo, por lo cual para estos autores no existe nada intrínsecamente anormal, los hitos madurativos se cumplen en el orden esperado, sólo que en forma más lenta. Esto implica que, a largo plazo, el niño adquirirá las habilidades deficitarias y siempre seguirá un orden específico en la adquisición de estas.
			Por lo anteriormente mencionado, el niño con retrasos en su desarrollo puede normalizarse a largo plazo y, cuando esto no ocurre, será diagnosticado con una cierta patología. Narbona y Schlumberger “contemplaron las diferentes posibilidades diagnósticas en las que puede desembocar un cuadro que inicialmente se manifestó como un retraso psicomotor de la siguiente manera: puede ocurrir que el retraso sea una variante normal del desarrollo, en cuyo caso se normalizará espontáneamente antes de la edad preescolar”. Puede que en realidad sea un verdadero retraso, debido a déficit en la estimulación por parte del entorno familiar y social, que podría ser normalizado si se adecuara la educación y el ambiente del niño (retraso de etiología ambiental); o bien deberse a enfermedad crónica extraneurológica (cardiopatía congénita, enfermedad respiratoria, desnutrición, entre otras), compensándose en la medida en que mejora la enfermedad general de base. Por otra parte, un retraso puede deberse al efecto de un déficit sensorial aislado, como la sordera neurosensorial congénita o ser la primera manifestación de una futura deficiencia mental, cuyo diagnóstico definitivo en los casos leves, no suele evidenciarse hasta el final de la edad preescolar. Otra posibilidad es que sea la primera manifestación de una encefalopatía crónica no evolutiva, un trastorno neuromuscular congénito de escasa o nula evolutividad, la primera manifestación de una futura torpeza selectiva en la psicomotricidad fina y/o gruesa (trastorno del desarrollo de la coordinación, frecuentemente asociado a la forma disatencional del TDAH), o el inicio de un trastorno global del desarrollo (trastorno de tipo autista).\\
			A veces es relativamente sencillo percibir si el retraso puede ser transitorio o no. En los casos en que los retrasos están asociados a otros signos o características físicas o dismorfias, por ejemplo, es más frecuente que se trate de un cuadro que tienda a mantenerse en el tiempo. Lo mismo ocurre en el retraso global del desarrollo donde hay alteración de dos o más áreas o campos del desarrollo, manifestándose un retraso significativo, correspondiente a dos o más desviaciones estándar inferior a la media en pruebas acorde a la edad del niño. Algunos ejemplos de trastornos globales del desarrollo son el autismo, el síndrome de Asperger o el síndrome de Rett. Cuando el problema del desarrollo es leve o sutil, puede no ser fácilmente evidenciable y para detectarlo es necesario realizar una prueba de pesquisa o screening. En estos casos podría hablarse de trastornos inaparentes del desarrollo psicomotor. Dado que la mayoría de los lactantes y preescolares con dificultades del desarrollo no tienen signos obvios de enfermedad, por lo menos en un inicio, ni factores de riesgo que lo sugieran, la identificación de estos niños aparentemente sanos suele constituir un verdadero desafío. Los trastornos inaparentes del desarrollo plantean tal vez la discusión más difícil en esta área y transcurren en un límite difuso entre lo "normal y patológico".\\
	\subsection{Marco Conceptual}
		Interfaz de usuario: La interfaz de usuario es el medio con que el usuario puede comunicarse con una máquina, un equipo o una computadora, y comprende todos los puntos de contacto entre el usuario y el equipo. Normalmente suelen ser fáciles de entender y fáciles de accionar. Las interfaces básicas de usuario son aquellas que incluyen elementos como menús, ventanas, teclado, ratón, los beeps y algunos otros sonidos que la computadora hace, y en general, todos aquellos canales por los cuales se permite la comunicación entre el ser humano y la computadora. La mejor interacción humano-máquina a través de una adecuada interfaz (Interfaz de Usuario), que le brinde tanto comodidad, como eficiencia.
		Tipos de interfaces de usuario: Dentro de las Interfaces de Usuario se puede distinguir básicamente tres tipos:
		Una interfaz de hardware, a nivel de los dispositivos utilizados para ingresar, procesar y entregar los datos: teclado, ratón y pantalla visualizadora.
		Una interfaz de software, destinada a entregar información acerca de los procesos y herramientas de control, a través de lo que el usuario observa habitualmente en la pantalla.
		Una interfaz de Software-Hardware, que establece un puente entre la máquina y las personas, permite a la máquina entender la instrucción y a el hombre entender el código binario traducido a información legible.
		Funciones principales: Sus principales funciones son las siguientes:
		Puesta en marcha y apagado.
		Control de las funciones manipulables del equipo.
		Manipulación de archivos y directorios.
		Herramientas de desarrollo de aplicaciones.
		Comunicación con otros sistemas.
		Información de estado.
		Configuración de la propia interfaz y entorno.
		Intercambio de datos entre aplicaciones.
		Control de acceso.
		Sistema de ayuda interactivo.
		Tipos de interfaces de usuario
		Según la forma de interactuar del usuario
		Atendiendo a como el usuario puede interactuar con una interfaz, nos encontramos con varios tipos de interfaces de usuario:
		Interfaces alfanuméricas (intérpretes de comandos) que solo presentan texto.
		Interfaces gráficas de usuario (GUI, graphic user interfaces), las que permiten comunicarse con el ordenador de una forma muy rápida e intuitiva a representando gráficamente los elementos de control y medida.
		Interfaces táctiles, que representan gráficamente un "panel de control" en una pantalla sensible que permite interactuar con el dedo de forma similar a si se accionara un control físico.
		Según su construcción Pueden ser de hardware o de software:
		Interfaces de hardware: Se trata de un conjunto de controles o dispositivos que permiten que el usuario intercambie datos con la máquina, ya sea introduciéndolos (pulsadores, botones, teclas, reguladores, palancas, manivelas, perillas) o leyéndolos (pantallas, diales, medidores, marcadores, instrumentos).
		Interfaces de software: Son programas o parte de ellos, que permiten expresar nuestros deseos al ordenador o visualizar su respuesta.
		Redes Neuronales: “Las redes neuronales también conocidas como sistemas conexionistas son un modelo computacional basado en un gran conjunto de unidades neuronales simples (neuronas artificiales) de forma aproximadamente análoga al comportamiento observado en los axones de las neuronas en los cerebros biológicos. La información de entrada atraviesa la red neuronal (donde se somete a diversas operaciones) produciendo unos valores de salida.
		Cada neurona está conectada con otras a través de unos enlaces. En estos enlaces el valor de salida de la neurona anterior es multiplicado por un valor de peso. Estos pesos en los enlaces pueden incrementar o inhibir el estado de activación de las neuronas adyacentes. Del mismo modo, a la salida de la neurona, puede existir una función limitadora o umbral, que modifica el valor resultado o impone un límite que se debe sobrepasar antes de propagarse a otra neurona. Esta función se conoce como función de activación.
		Estos sistemas aprenden y se forman a sí mismos, en lugar de ser programados de forma explícita, y sobresalen en áreas donde la detección de soluciones o características es difícil de expresar con la programación convencional. Para realizar este aprendizaje automático, normalmente, se intenta minimizar una función de pérdida que evalúa la red en su total. Los valores de los pesos de las neuronas se van actualizando buscando reducir el valor de la función de pérdida. Este proceso se realiza mediante la propagación hacia atrás.
		El objetivo de la red neuronal es resolver los problemas de la misma manera que el cerebro humano, aunque las redes neuronales son más abstractas. Los proyectos de redes neuronales modernos suelen trabajar desde unos miles a unos pocos millones de unidades neuronales y millones de conexiones que, si bien son muchas órdenes, siguen siendo de una magnitud menos compleja que la del cerebro humano, más bien cercana a la potencia de cálculo de un gusano.
		Nuevas investigaciones sobre el cerebro a menudo estimulan la creación de nuevos patrones en las redes neuronales. Un nuevo enfoque está utilizando conexiones que se extienden mucho más allá y capas de procesamiento de enlace en lugar de estar siempre localizado en las neuronas adyacentes. Otra investigación está estudiando los diferentes tipos de señal en el tiempo que los axones se propagan, como el aprendizaje profundo, interpola una mayor complejidad que un conjunto de variables booleanas que son simplemente encendido o apagado.
		Las redes neuronales se han utilizado para resolver una amplia variedad de tareas, como la visión por computador y el reconocimiento de voz, que son difíciles de resolver usando la ordinaria programación basado en reglas. Históricamente, el uso de modelos de redes neuronales marcó un cambio de dirección a finales de los años ochenta de alto nivel, que se caracteriza por sistemas expertos con conocimiento incorporado en si-entonces las reglas, a bajo nivel de aprendizaje automático, caracterizado por el conocimiento incorporado en los parámetros de un modelo cognitivo con algún sistema dinámico."
	\subsection{Marco Espacial}
	\subsection{Marco Legal}
\section{Metodología de la investigación}
	\subsection{Tipo de estudio}
		Para llevar a cabo el desarrollo del prototipo es necesario realizar una búsqueda conceptual y práctica de la forma en que se recolecta la información de cada niño y analizar también la manera en que se realiza la elección de tratamientos para resolver cierta dificultad en los niños, a su vez analizar como los modelos predictivos pueden dar un uso adecuado a los tratamientos a utilizar, por lo cual el tipo de estudio es descriptivo, basado en la búsqueda de disminuir la dependencia manual en la asignación de tratamientos.
		\subsection{Método de investigación}
		Para realizar el análisis necesario, se utilizarán los siguientes métodos de investigación:
		\begin{itemize}
			\item Método de observación como procedimiento en los diagnósticos y tratamientos como problemáticas en la investigación, logrando obtener como base información que contextualice el problema.
			\item Método analítico para obtener variables y datos que logran identificar la problemática de la investigación. Estos llevan un análisis exhaustivo que permite obtener la solución del problema.
		\end{itemize}
	\subsection{Fuentes y técnicas para la recolección de la información}
		Como fuentes de información se tienen institutos prestadores de salud (IPS) en los cuales se realizan procesos de integración sensorial en niños para mejorar su desarrollo, también terapeutas ocupacionales independientes las cuales tienen sitios de atención con los equipos que usan para los mismos, también fuentes como videos y ayudas directas de los profesionales que realizan esta clase de tratamientos.
	\subsection{Tratamiento de la información}
\section{Organización del trabajo de grado}
\section{Estudio de sistemas previos}

    \part{Desarrollo de la Investigación}
    	\section{Ejecución del proyecto}
    \section{Diseño de Arquitectura de Negocio}
      \subsection{Punto de Vista Organizaciónal}
      El punto de vista de la organización se enfoca en el interior de la organización, un departamento, una red de empresas, es un punto de vista muy útil ya que permite identificar  competencias, autoridad y responsabilidades en una organización. \cite{ref9}
  \begin{table}[h]
	\centering
	\begin{tabular}{p{3.7cm}p{8cm}}
		\hline
        \textbf{Nombre} & \textbf{Organización} \\
		\hline
		\textbf{Stakeholders} & Organización, arquitectos de dominio y proceso, gerentes, empleados, accionistas.\\
		\textbf{Preocupaciones}v & Identificación de competencias, autoridad y responsabilidades.\\
		\textbf{Propósito} & Diseñar, decidir, informar. \\
		\textbf{Nivel de Abstracción} & Coherencia. \\
		\textbf{Capa} & Capa de negocio. \\
		\textbf{Aspectos} & Activo \\
	\end{tabular}
	\caption{Descripción Punto de Vista de la Organización \cite{ref9}}
	\label{tabla4}
  \end{table}
  \begin{figure}[h]
 	\centering
 	\includegraphics[scale=0.2]{Imagenes/Figuras/14.png}
 	\caption{Posición del punto de vista de organización conceptualmente y marco del punto de vista \cite{ref9}}
 	\label{figura14}
  \end{figure}
      \subsubsection{Metamodelo Punto de Vista Organizaciónal}
      La Figura \ref{metamodelo6} ilustra el punto de vista de producto el cual es la convergencia de los puntos de vista anteriores, es el esfuerzo por conocer la estructura, el esfuerzo por saber qué hace cada persona todo converge en el punto de vista que apunta al producto, el cual es un conjunto de servicios al cual se le adhiere un contrato y como elemento clave se le destaca un valor; el producto reposa sobre los procesos que son hechos por unos roles de negocio los cuales corresponden a unos actores. \cite{ref9}
  \begin{figure}[h]
	\centering
	\includegraphics{Imagenes/Metamodelos/01.png}
	\caption{Metamodelo Punto de Vista de Producto \cite{ref9}}
	\label{metamodelo6}
  \end{figure}
      \subsubsection{Aplicación Punto de Vista Organizaciónal}
      \subsection{Punto de Vista de Cooperación}
      Este punto de vista se enfoca en los actores y sus relaciones con el entorno que los cobija, es un punto de vista donde los Stakeholders o interesados son la organización, los procesos y los arquitectos de dominio. La finalidad u objetivo de este punto de vista es la de diseñar, decidir e informar. \cite{ref9}

      \begin{table}[h]
      	\centering
      	\begin{tabular}{p{3.7cm}p{8cm}}
      		\hline
      		\textbf{Nombre} & \textbf{Cooperación} \\
      		\hline
      		\textbf{Stakeholders} & Organización, arquitectos de dominio y proceso \\
      		\textbf{Preocupaciones} & Relación de actores con el entorno \\
      		\textbf{Propósito} & Diseñar, decidir, informar \\
      		\textbf{Nivel} de Abstracción & Detalle \\
      		\textbf{Capa} & Capa de negocio \\
      		\textbf{Aspectos} & Estructura Activa, Comportamiento \\
      	\end{tabular}
      	\caption{Descripción Punto de Vista Cooperación de Actor \cite{ref9}}
      	\label{tabla5}
      \end{table}

      \begin{figure}[h]
      	\centering
      	\includegraphics[scale=0.2]{Imagenes/Figuras/15.png}
      	\caption{Posición del punto de vista cooperación de actor conceptualmente y marco del punto de vista \cite{ref9}}
      	\label{figura15}
      \end{figure}
      \subsubsection{Metamodelo Punto de Vista Cooperación}
      En la Figura \ref{metamodelo2} se ilustra el metamodelo perteneciente al punto de vista de cooperación de actor el cual esta compuesto de los conceptos de actor, rol, interface, colaboración, servicio de negocio, servicio de aplicación, interface de comunicación de aplicación y componentes de aplicación.\\

      En este punto de vista a el actor se le asigna un rol, el rol se compone de interfaces, a la interface se le asigna servicios y estos servicios van a una capa de aplicación a través de una interface que está conformada por componentes de aplicación y es usada por componentes de aplicación. \\

      Otro uso importante del punto de vista de cooperación de actor es mostrar como un número de actores de negocio cooperantes y / o componentes de aplicación juntos realizan un proceso de negocio. Por lo tanto, en esta vista, tanto actores de negocio como roles y componentes de aplicación pueden aparecer. \cite{ref9}

      \begin{figure}[h]
      	\centering
      	\includegraphics{Imagenes/Metamodelos/02.png}
      	\caption{Metamodelo Punto de Vista Cooperación de Actor \cite{ref9}}
      	\label{metamodelo2}
      \end{figure}
      \subsubsection{Aplicación Punto de Vista Cooperación}
      \section{Punto de Vista Función de Negocio}
      En este punto de vista se muestran las principales funciones de negocio de la organización y sus relaciones en términos de los flujos de información, valor, o productos entre ellas, los Stakeholders o interesados son los procesos y arquitectos de dominio, administradores operacionales, tiene especial cuidado en la estructura de los procesos de negocio, su coherencia, integridad y las responsabilidades. \cite{ref9}

      \begin{table}[h]
      	\centering
      	\begin{tabular}{p{3.7cm}p{8cm}}
      		\hline
      		\textbf{Nombre} & \textbf{Función de Negocio}\\
      		\hline
      		\textbf{Stakeholders} & Organización, arquitectos de dominio y proceso \\
      		\textbf{Preocupaciones} & Identificación de competencias, identificación de actividades principales, reducción de la complejidad \\
      		\textbf{Propósito} & Diseñar \\
      		\textbf{Nivel de Abstracción} & Coherencia \\
      		\textbf{Capa} & Capa de negocio \\
      		\textbf{Aspectos} & Comportamiento (Activo) \\
      	\end{tabular}
      	\caption{Descripción Punto de Vista Función de Negocio}
      	\label{Tab:tabla6}
      \end{table}

      \begin{figure}[h]
      	\centering
      	\includegraphics[scale=0.2]{Imagenes/Figuras/16.png}
      	\caption{Posición del punto de vista función de negocio conceptualmente y marco del punto de vista \cite{ref9}}
      	\label{figura16}
      \end{figure}

      \subsection{Metamodelo}
      La Figura \ref{metamodelo3} muestra los conceptos de actor, rol y función; aquí se asignan tareas o funciones a estos actores, en el modelo se extrae lo que interesa, lo que se quiere capturar o tener en la mente. En este metamodelo aparecen dos tipos de relaciones el flujo y los disparos. \cite{ref9}

      \begin{figure}[h]
      	\centering
      	\includegraphics{Imagenes/Metamodelos/03.png}
      	\caption{Metamodelo}
      	\label{metamodelo3}
      \end{figure}

      \section{Punto de Vista Proceso}
      El punto de vista de proceso de negocio es el encargado de mostrar una estructura de alto nivel y composición de uno o más procesos de negocio. Tiene una complejidad importante, se incorporan elementos de comportamiento se incluye el proceso y/o función de negocio como elemento central, el proceso y/o función de negocio se ve afectado por las mismas relaciones con los demás conceptos, este punto de vista llama la atención en que nos induce a las entrañas de las organizaciones porque se ve lo que ellas hacen. \cite{ref9}

      \begin{table}[h]
      	\centering
      	\begin{tabular}{p{3.7cm}p{8cm}}
      		\hline
      		\textbf{Nombre} & \textbf{Proceso} \\
      		\hline
      		\textbf{Stakeholders} & Arquitectura de dominio y proceso, Gerentes de operación \\
      		\textbf{Preocupaciones} & Estructurar los procesos del negocio, consistencia, integridad y responsabilidades \\
      		\textbf{Propósito} & Diseñar \\
      		\textbf{Nivel de Abstracción} & Detalle \\
      		\textbf{Capa} & Capa de negocio \\
      		\textbf{Aspectos} & Comportamiento (Activo), (Pasivo) \\
      	\end{tabular}
      	\caption{Descripción punto de vista proceso de negocio \cite{ref9}}
      	\label{Tab:tabla7}
      \end{table}

      \begin{figure}[h]
      	\centering
      	\includegraphics[scale=0.2]{Imagenes/Figuras/17}
      	\caption{Posición del punto de vista proceso de negocio conceptualmente y marco del punto de vista \cite{ref9}}
      	\label{figura17}
      \end{figure}

      \subsection{Metamodelo}
      La Figura \ref{metamodelo4} se aprecia que el proceso de negocio tiene que ver con un rol o conjunto de roles, el proceso de negocio es disparado por un evento y el proceso de negocio genera un evento o un proceso de eventos, los procesos no son máquinas infinitas todo proceso es
      iniciado por uno o un conjunto de eventos.

      Los procesos generan objetos de negocio que es la representación del trabajo en la organización,  el servicio de negocio es lo que el proceso de negocio lleva a cabo estableciéndose entre los dos una relación de realización, el servicio es el core de negocio lo que el cliente mira, el proceso de negocio es lo que implementa realiza, materializa el servicio y el proceso de negocio funciona porque existen unos roles que se encargan de realizar el proceso. \cite{ref9}

      \begin{figure}[h]
      	\centering
      	\includegraphics{Imagenes/Metamodelos/04.png}
      	\caption{Metamodelo}
      	\label{metamodelo4}
      \end{figure}

      \section{Punto de Vista Cooperación}
      El punto de vista de cooperación de proceso de negocio es usado para mostrar las relaciones
      de uno o mas procesos de negocio con los demás procesos de negocio y / o con su ambiente. Puede ser usado tanto para crear un diseño de alto nivel de procesos de negocio dentro de su contexto como para proveer un responsable administrador operacional para uno o mas de tales procesos con mando en sus dependencias. \cite{ref9}

      \begin{table}[h]
      	\centering
      	\begin{tabular}{p{3.7cm}p{8cm}}
      		\hline
      		\textbf{Nombre} & \textbf{Cooperación} \\
      		\hline
      		\textbf{Stakeholders} & Arquitectos de domino, Gerentes de Operaciones \\
      		\textbf{Preocupaciones} & Dependencias de los procesos de negocio, Responsabilidades \\
      		\textbf{Propósito} & Diseñar, decidir \\
      		\textbf{Nivel} de Abstracción \\
      		\textbf{Capa} & Capa de Negocio) \\
      		\textbf{Aspectos} & Comportamiento, (activo), (pasivo) \\
      	\end{tabular}
      	\caption{Descripción punto de vista de cooperación de proceso \cite{ref9}}
      	\label{tabla8}
      \end{table}

      \begin{figure}[h]
      	\centering
      	\includegraphics[scale=0.2]{Imagenes/Figuras/18.png}
      	\caption{Posición del punto de vista de cooperación de proceso conceptualmente y marco del punto de vista \cite{ref9}}
      	\label{figura18}
      \end{figure}

      \subsection{Metamodelo}
      En la Figura \ref{metamodelo5} se ilustra el metamodelo perteneciente al punto de vista de cooperación de proceso el cual esta compuesto de los conceptos de los procesos del negocio y sus responsabilidades. En este punto de vista a al proceso se le asigna un rol, el rol se compone de interfaces, a la interface se le asigna interacciones y estas interacciones van a una capa de aplicación a través de una interface. \cite{ref9}

      \begin{figure}[h]
      	\centering
      	\includegraphics{Imagenes/Metamodelos/05.png}
      	\caption{Metamodelo}
      	\label{metamodelo5}
      \end{figure}

      \section{Punto de Vista de Producto}
      Este punto de vista se describe como eje central el valor que uno o más productos ofrecen a la clientes u otras partes externas involucradas con la organización, muestra además la composición de uno o más productos en términos de cómo están compuestos, la asociación, el contrato y otros acuerdos. El punto de vista del producto se suele utilizar en el desarrollo de productos para diseñar un producto componiendo servicios existentes o mediante la identificación de nuevos servicios que se tienen que crear para este producto, dado el valor que un cliente espera de ella. \cite{ref9}

      \begin{table}[h]
      	\centering
      	\begin{tabular}{p{3.7cm}p{8cm}}
      		\hline
      		\textbf{Nombre} & \textbf{Vista de Producto} \\
      		\hline
      		\textbf{Stakeholders} & Diseñadores de producto, gerentes de producto, Arquitectos de proceso y de dominio \\
      		\textbf{Preocupaciones} & Desarrollo del producto y el valor que este ofrece a la organización \\
      		\textbf{Propósito} & Diseñar, decidir \\
      		\textbf{Nivel de Abstracción} & Coherencia \\
      		\textbf{Capa} & Capa de Negocio \\
      		\textbf{Aspectos} & Comportamiento, información, (activo) \\
      	\end{tabular}
      	\caption{Descripción Punto de Vista de Producto}
      	\label{tabla9}
      \end{table}

      \begin{figure}[h]
      	\centering
      	\includegraphics[scale=0.2]{Imagenes/Figuras/19.png}
      	\caption{Posición del Punto de Vista de Producto}
      	\label{figura19}
      \end{figure}

      \subsection{Metamodelo}
      La Figura \ref{metamodelo6} ilustra el punto de vista de producto el cual es la convergencia de los puntos de vista anteriores, es el esfuerzo por conocer la estructura, el esfuerzo por saber qué hace cada persona todo converge en el punto de vista que apunta al producto, el cual es un conjunto de servicios al cual se le adhiere un contrato y como elemento clave se le destaca un valor; el producto reposa sobre los procesos que son hechos por unos roles de negocio los cuales corresponden a unos actores. \cite{ref9}

      \begin{figure}[h]
      	\centering
      	\includegraphics{Imagenes/Metamodelos/06.png}
      	\caption{Metamodelo}
      	\label{metamodelo6}
      \end{figure}
     \section{Diseño de Arquitectura de Aplicación}
      \subsection{Punto de Vista Comportamiento de Aplicación}
      \subsubsection{Metamodelo de Punto de Vista de Comportamiento de Aplicación}
      \subsubsection{Aplicación Punto de Vista de Comportamiento de Aplicación}
      \subsection{Punto de Vista Cooperación de Aplicación}
      \subsubsection{Metamodelo de Punto de Vista de Cooperación de Aplicación}
      \subsubsection{Aplicación Punto de Vista de Cooperación de Aplicación}
      \subsection{Punto de Vista Estructura de Aplicación}
      \subsubsection{Metamodelo de Punto de Vista de Estructura de Aplicación}
      \subsubsection{Aplicación Punto de Vista de Cooperación de Aplicación}
      \subsection{Punto de Vista Uso de Aplicación}
      \subsubsection{Metamodelo de Punto de Vista de Uso de Aplicación}
      \subsubsection{Aplicación Punto de Vista de Cooperación de Aplicación}
     \section{Diseño de Arquitectura de Infraestructura y Datos ˜}
      \subsection{Punto de Vista de Infraestructura}
      \subsubsection{Metamodelo de Punto de Vista de Infraestructura}
      \subsubsection{Aplicación Punto de Vista de Infraestructura}
      \subsection{Punto de Vista de Uso de Infraestructura}
      \subsubsection{Metamodelo de Punto de Vista de Uso de Infraestructura}
      \subsubsection{Aplicación Punto de Vista de Uso de Infraestructura}
      \subsection{Punto de Vista Organización e implementación}
      \subsubsection{Metamodelo de Punto de Vista de Organización e implementación}
      \subsubsection{Aplicación Punto de Vista de Organización e implementación}
      \subsection{Punto de Vista Estructura de Información}
      \subsubsection{Metamodelo de Punto de Vista de Estructura de Información}
      \subsubsection{Aplicación Punto de Vista de Estructura de Información}
      \subsection{Punto de Vista de Realización del Servicio}
      \subsubsection{Metamodelo de Punto de Vista de Realización del Servicio}
      \subsubsection{Aplicación Punto de Vista de Realización del Servicio}
      \subsection{Punto de Vista de Capas}
      \subsubsection{Metamodelo de Punto de Vista de Capas}
      \subsubsection{Aplicación Punto de Vista de Capas}
     \section{Diseño de Arquitectura Motivacional}
      \subsection{Punto de Vista de Stakeholder}
      \subsubsection{Metamodelo de Punto de Vista de Stakeholder}
      \subsubsection{Aplicación de Punto de Vista de Stakeholder}
      \subsection{Punto de Vista de Realización de Objetivos}
      \subsubsection{Metamodelo de Punto de Vista de Realización de Objetivos}
      \subsubsection{Aplicación de Punto de Vista de Realización de Objetivos}
      \subsection{Punto de Vista de Contribución}
      \subsubsection{Metamodelo de Punto de Vista de Contribución}
      \subsubsection{Aplicación Punto de Vista de Contribución}
      \subsection{Punto de Vista de Principios}
      \subsubsection{Metamodelo de Punto de Vista de Principios}
      \subsubsection{Aplicación Punto de Vista de Principios}
      \subsection{Punto de Vista de Realización de Requerimientos}
      \subsubsection{Metamodelo de Punto de Vista de Realización de Requerimientos}
      \subsubsection{Aplicación Punto de Vista de Realización de Requerimientos}
      \subsection{Punto de Vista de Motivación}
      \subsubsection{Metamodelo de Punto de Vista de Motivación}
      \subsubsection{Aplicación Punto de Vista de Motivación}
     \section{Diseño de Arquitectura de Proyecto}
      \subsection{Punto de Vista de Proyecto}
      \subsubsection{Metamodelo de Punto de Vista de Proyecto}
      \subsubsection{Aplicación de Punto de Vista de Proyecto}
      \section{Diseño de Arquitectura de Migración}
      \subsection{Punto de Vista de Migración}
      \subsubsection{Metamodelo de Punto de Vista de Migración}
      \subsubsection{Aplicación de Punto de Vista de Migración}
      \subsection{Punto de Vista de Migración e Implementación}
      \subsubsection{Metamodelo de Punto de Vista de Migración e Implementación}
      \subsubsection{Aplicación de Punto de Vista de Migración e Implementación}

    \part{Cierre de la Investigación}
	    \section{Resultados y Discusión}
\subsection{Resultados}
\section{Conclusiones}
\subsection{Verificación, contraste y evaluación de los objetivos}
\subsection{Síntesis del modelo propuesto}
\subsection{Aportes Originales}
\subsection{Trabajos o publicaciones derivadas}
\section{Prospectiva del trabajo de grado}
\subsection{Líneas de investigación futuras}
\subsection{Trabajos de Investigación futuros}

\end{document}
